% ex: tw=74 ts=2 sw=2 noet sts=2
\documentclass[10pt]{article}
\usepackage[utf8]{inputenc}
\usepackage{amsmath}
\usepackage{xkeyval}
\usepackage[bookmarksnumbered,frenchlinks]{hyperref}
%\hypersetup{pdfborder=0 0 0}
\usepackage{multirow}
\usepackage[english]{babel}
\usepackage{fullpage}
\usepackage{tabulary}
\usepackage{tabularx}
%\usepackage{natbib}
\usepackage[all]{hypcap}
\usepackage{hyperref}
\usepackage{framed}
\usepackage{fullpage}
\usepackage{graphicx}
\usepackage{listings}
\usepackage{subfig}
\usepackage{verbatim}
\usepackage{float}
\usepackage{datatool}
%\floatstyle{boxed} 
%\restylefloat{figure}

\title{\textbf{Homework 5}\\
Math Filesystem via FUSE}

\author{Michael Koval and Cody Schafer}
\date{\today}
\begin{document}
\maketitle

\section{Project}

Impliment a filesystem that supports a range of mathematical operations.
Each component of the path is either an operation of a number to be
operated on. Additionally, at the top level the functions are listed.
Documentation is supplied for each math operation within the ``doc'' file
which is in the directory of the operation.

\section{Implementation}

A parser was written with decomposes each path into a linked list of OPs
(operations), NUMs (numbers), and UNKs (unknowns). Functions then operate
on this doubly linked list (also known as the ``parse list'' or ``plist'')
rather than using ad-hoc parsing solutions strewn throughout the codebase.

The parser fully supports nesting of functions and multiple return values
from the evaluation of a ``path program'' (the data returned after all
OPs within a path are evaluated). This is implimented via a postfix
evaluated stack-style interface, the plist is iterated over in reverse,
and each operation pulls numbers from the elements of the plist that have
already been considered for evaluation. The OPs then push back onto the
stack the value(s) they wish to return.

For the fuse interface, objects in the filesystem cannot be said to be
both files and directories, meaning some way to determine when a value is
desired in the call to ``getattr'' is needed. To do this, we assume that
once the ``path program'' successfully evaluates, that value is desired.
This eliminates the posibility of path programs such as
\texttt{/add/3/2/add/1/2 $\larrow$ 5, 3}, however this loss is not
considered significant. Basic nesting is still maintained.




\section{Further Possibilities}

\end{document}
