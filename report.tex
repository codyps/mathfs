%
% Cody Schafer <cpschafer@gmail.com>
% Michael Koval <koval.michael@gmail.com>
%
% ex: tw=74 ts=2 sw=2 noet sts=2
\documentclass[10pt]{article}
\usepackage[utf8]{inputenc}
\usepackage{amsmath}
\usepackage{xkeyval}
\usepackage[bookmarksnumbered,frenchlinks]{hyperref}
%\hypersetup{pdfborder=0 0 0}
\usepackage{multirow}
\usepackage[english]{babel}
\usepackage{fullpage}
\usepackage{tabulary}
\usepackage{tabularx}
%\usepackage{natbib}
\usepackage[all]{hypcap}
\usepackage{hyperref}
\usepackage{framed}
\usepackage{fullpage}
\usepackage{graphicx}
\usepackage{listings}
\usepackage{subfig}
\usepackage{verbatim}
\usepackage{float}
\usepackage{datatool}

\title{MathFS}

\author{
	Michael Koval \\
	mkoval@eden.rutgers.edu
	\and
	Cody Schafer \\
	cschafer@eden.rutgers.edu
}
\date{\today}
\begin{document}
\maketitle

\section{Project Overview}
The final goal of this project was to a file system that supports
on-the-fly calculation of several pre-defined mathematical operations. The
root directory in the file system contains one directory for each
supported mathematical function. Each of these directories contains a
visible documentation file, \texttt{doc}, and a potentially infinite
number of implicit files and/or sub-directories. These files or
sub-directories are either the operation's arguments (e.g.
\texttt{add/1/2}) numbers), or a nested operation (e.g.
\texttt{add/1/sub/5/1}). In the case of multiple functions, the path is
evaluated using \textit{prefix notation}.

\section{Implementation}
A parser was written with decomposes each path into a doubly linked
list, known as a \textit{parse list}, where each node is either an
operation, number, or unknown. Tokenizing the string in one step isolates
the string parsing from the file system logic and simplified the
implementation of nested functions with multiple return values.

Evaluating nested functions was implemented by using a prefix expression
parser that iterates over the parse list in reverse order. Every number is
pushed onto the stack until an operation is encountered. The operation
pops the necessary number of arguments off of the stack, performs the
requested computation, and pushes its return value(s) back onto the stack.
All numbers that remain on the stack after the parse is complete are
printed as output.

Unfortunately, it is not straightforward to merge this parser with a FUSE
interface because no object in the file system can simultaneously be a file
and a directory. Even worse, a valid implementation of \textit{getattr()}
must unambiguously identify each part of a path as either a file or
directory. This problem was avoided by the following assumption: a path
points to a file if and only if it can be evaluated with no errors. If
there is an error, it is assumed to be the beginning of a path with
additional arguments. This renders paths similar to
\texttt{/add/3/2/add/1/2} $\rightarrow \{5, 3\}$ invalid, but does not
impact any common uses of nesting.

\section{Additional Notes}
Due to extensive use of \textit{typedef}s and pre-processor macros, the
internals of MathFS are mostly type-agnostic. The integral (\texttt{long})
math operations can be replaced with their floating-point
(\texttt{double}) equivalents by building the program with the
\texttt{NUM\_FLOAT} pre-processor variable defined: \texttt{make clean
\&\& make CFLAGS=-DNUM\_FLOAT}. Note that the \texttt{factor} operation is
no longer valid and will be removed at compile time.

\end{document}
